\chapter{Fundamentos tecnologícos}
\label{chap:Fundamentos tecnologícos}

\section{Vue}

Vue es un framework de JavaScript que permite crear potentes y versátiles interfaces de usuario. Una de las grandes ventajas que tiene vue es que combina HTML, CSS y JavaScript en un mismo fichero, permitiéndonos tener más ordenado y localizable nuestro código. 

Este framework nos permite crear componente que después van poder ser reutlizados por otros componentes, lo que nos va dar una gran ventaja a la hora de ahorrar tiempo y costes en el proyecto.
Cada vez que hagamos un cambio en alguno de nuestro componente, este se actualiza automáticamente en la interfaz gracias a su propiedad reactiva.

Incluye algunas herramientas, como Vue Router, para gestionar el enrutamiento de las páginas que lo utilizaremos en el proyecto para nuestra interfaz.

Respecto a otros frameworks como Angular o React, vue nos proporcionan una serie de ventajas:

• \textbf{Curva de aprendizaje suave:} Es simple y fácil de aprender. Contiene bastante documentación clara, lo que permite desarrollar código de forma rápida.

• \textbf{Flexibilidad} respecto a frameworks como Angular, Vue permite al desarrollador elegir y configurar las partes que este considere oportunas únicamente.

• \textbf{Tamaño ligero:} Comparado con los otros frameworks, Vue.js es relativamente pequeño, lo que puede resultar de gran ayuda a la hora de tiempos de carga y rendimiento.

• \textbf{Es muy parecido a HTML}, por lo que la sintaxis es muy fácil de entender si ya trabajaste anteriormente con este. 


Utilizaremos este framework junta a Tailwindcss para el desarrollo de nuestra interfaz.

\section{TailwindCSS}

Tailwindcss es un framework de CSS que nos permite componer estilos de manera mucho más sencilla. Nos permite relacionar y crear clases css para utilizar en nuestros elementos HTML fácilmente.
	
A diferencia de Bootstrap o foundation, tailwind lo que nos proporcionan son conjunto de clases css de bajo nivel que son fácilmente combinables para así crear diseños sin la necesidad de escribir CSS personalizado.
	
Gracias a esta característica, nos proporciona un desarrollo de código más rápido al eliminar la necesidad de escribir todo el CSS personalizado para cada componente. 
	
Gracias al tener una clases de utilidad comunes, esto nos permite que el diseño en nuestra aplicación sea fácilmente mantenible y no preocuparnos de tener conflictos con los css.
	
Una clave en todo esto es que Tailwind no impone estilos predeterminados ni componentes específicos, lo que nos da una gran libertad a la hora de crear los diseños.
	
Por eso se eligio Tailwind, por su gran facilidad, flexibilidad y personalización a la hora de crear nuestro proyecto. 

\section{Shadcn-Vue}

Shadcn es una biblioteca de componentes que permite una fácil personalización de estos y de su estilo. Esto nos permite construir nuestra aplicación de forma eficiente y manejable, al utilizar componentes reutilizables.
	
La versión que utilizaremos en el proyecto es shadcn junto a vue o shadcn-vue, que permite una mayor facilidad a la hora de incluir los componentes, ya que estos están creados en vue.js.

Esto también nos permite que a la hora de añadir un componente, este se actualice automáticamente en la interfaz y si hacemos algún cambio en él, también se añada de forma correcta. 

\section{Visual Studio Code}

Visual Studio Code es un editor de código fuente desarrollado por Microsoft. Es conocido por ser ligero, rápido y extensible. Es gratuíto y de código abierto, lo que nos permite tener un gran variedad de extensiones para añadir a nuestro código.
	
Gracias a su gran variedad de lenguajes soportados, se eligió este ide ya que soporta Vue.js, por lo que es perfecto para desarrollar nuestra interfaz. También se eligió porque nos permite una serie de característica como son:
	
• \textbf{Es ligero y rápido}, ya que como vue.js, este tiene un rendimiento eficiente y tiempos de carga rápidos, lo que nos permite un desarrollo más eficiente.

• Tiene \textbf{miles de extensiones disponibles} y muchas muy útiles para nuestro proyecto.

• \textbf{Ofrece itelliSense}, que nos autocompleta el código para así poder avanzar rápidamente con nuestra funciones y métodos.

• Contiene una \textbf{terminal integrada}, que permite ejecutar comandos en línea para añadir librerías o componentes de las bibliotecas que se nombraron anteriormente.
