\chapter{Metología}

En este apartado se detalla la metodología utilizada, así como las razones porque se seleccionaron esta metodología respecto a otras


\section{Metodología eligida}

Para la realización de este proyecto se seguirá una metodología de desarrollo ágil basada en SCRUM.

Esta se adaptará al proyecto para obtener los mejores resultados de la manera más práctica posible, incluyendo un desarrollo incremental a través de iteraciones en las que definen los requisitos mínimos de funcionalidades, que se desarrollaran, se documentaran y se validarán antes de pasar a la siguiente iteración. Se mantienen reuniones semanales para definir y evaluar el grado de cumplimiento de los objetivos de cada iteración, revisar la planificación y detectar y corregir problemas que dificulten el avance del proyecto.

Gracias al seguir esta metodología nos proporcionan una serie de ventajas respecto a otras como son:

• Resultados anticipados: No es necesario esperar al final para obtener resultados del proyecto. Al cierre de cada semana de trabajo, ya se pueden ver ciertos logros, lo que permite probar el producto sin estar finalizado.

• Flexibilidad y adaptación: Scrum nos permite adaptarnos según las prioridades o los cambios que queremos.  Al final de cada iteración, se puede hacer pruebas de concepto sobre el producto para así tomar las mejores decisiones en función del resultado que obtenemos, permitiéndonos hacer cambios para solucionar los problemas que vayan apareciendo en el proyecto.

• Gestión de las expectativas del usuario: Los usuarios pueden participar en cada una de las etapas del proceso y proponer soluciones.  Pueden aportar ideas en cualquier parte del proyecto para determinar soluciones a los problemas que vayan apareciendo.

• Gestión sistemática de riesgos: SCRUM nos permite que los problemas que vayan apareciendo, puedan ser gestionados en el momento de su aparición. La intervención será inmediata a la hora de que aparezca un riesgo. Como el trabajo se divide en iteraciones, la cantidad de riesgo a los que nos vamos a enfrentar es menor y esta limitado, lo que ayuda a la complejidad del proyecto.

