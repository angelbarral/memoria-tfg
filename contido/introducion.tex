\chapter{Introdución}
\label{chap:introducion}

\section{Contexto}

\lettrine Cualquier persona de una empresa pierde gran parte de su tiempo para ser capaz de acceder a la información 
que necesita para llevar su trabajo o en registrar nuevos datos en un sistema de información de una empresa. 

Para solucionar esto tenemos SINVAD, un sistema de compresión de lenguaje natural que es desarrollado por la empresa SREC Solutions. 
El objetivo es asistir al usuario para consultar información, registrar datos y supervisar información desde cualquier dispositivo con sus propias palabras, 
en múltiples idiomas y adaptado a los procesos de empresa.

Pero esto tiene una gran limitación, y es que el cliente debe construir sus propias aplicaciones para integrar la API REST de 
SINVAD para así poder generar informes a partir de los registros de los históricos del cliente.

Por tanto, el objetivo de este proyecto es generar una interfaz interactiva de navegación de históricos asociados que 
facilite la integración de la API de SINVAD y nos permita gestionar la generación, consulta, interacción, edición en tiempo real, 
descarga y generación de notificaciones en tiempo real de estos informes que se nombraron anteriormente.

Se busca que esta interfaz sea fácil de utilizar y intuitiva, para que cualquiera persona de cualquiera edad sepa utilizarla, 
y mediante unas pocas y sencillas acciones, sea capaz de generar informes sobre cualquiera de los datos que tienen el histórico.

\section{Objetivos}

En relación con lo anterior, a continuación se detallarán los objetivos que debería cumplir nuestra interfaz.

Esta interfaz debe estar disponible vía web y cualquier usuario que tenga acceso tiene que ser capaz de utilizarla de forma sencilla.
Para esto buscaremos que nuestra interfaz sea intuitiva y con esto nos referimos que cualquiera persona, con o sin conocimientos tecnologicos,
sea capaz mediante una serie sencilla de pasos, sacar los informes que a ellos le interesen.

Dentro de esta interfaz, el usuario podrá hacer los siguientes pasos:

- \textbf{Título del informe:} El usuario, antes de empezar a configurar la información que quiere que aparezca en el informe, necesitará indicar el propósito por el cuál se crea, y
esto se indicará mediante el título.

- \textbf{Periocidad del informe:} El usuario podrá indicar la periocidad con la que quiere generar el informe, ya que estos, se generarán automáticamente cada x tiempo.

- \textbf{Número de columnas:} El usuario podrá elegir el número de columnas que aparezcan en su informe y el contenido de cada una de ellas. Si alguna de estas, no muestra
el dato correcto, se podrá editar antes de generar el informe.

- \textbf{Contenido del informe: } Al seleccionar el número de columnas, aparte del título de estas, será necesario el dato a buscar mediante una expresión de lenguaje natural, y 
esto será lo que el usuario le pasará a nuestro LLM para que lo procese.

Para cumplir todos estos pasos, la interfaz los contendrá de forma sencilla y intuitiva como se mostrará mas adelante en este documento, para facilitar al usuario
este proceso lo máximo posible.

